\documentclass[Portugues]{ic-tese-v3}
%\documentclass[10pt]{article}
\usepackage{graphicx}
\usepackage{float}
\usepackage{amsmath}
\usepackage{amsfonts}
\usepackage[utf8]{inputenc}
%\usepackage[backend=biber, sorting=none]{biblatex}
\usepackage{csquotes}
\usepackage{docmute}
\usepackage{ragged2e}
\usepackage[T1]{fontenc}
\usepackage{complexity}
\usepackage{graphicx,url}
\usepackage{psfrag}
\usepackage{multirow}
\usepackage{amsmath}
\usepackage[normalem]{ulem}
\useunder{\uline}{\ul}{}
%\usepackage[brazilian,hyperpageref]{backref}	 % Paginas com as citações na bibl
%\usepackage[alf]{abntex2cite}

%Pacote extra para algoritmos
\usepackage[portuguese, ruled, linesnumbered]{algorithm2e}

\definecolor{darkgreen}{rgb}{0,0.5,0}
\definecolor{gray}{rgb}{0.34,0.34,0.36}
%Para colocar trechos de código
\usepackage{listings}
\lstset{
  language=Python,
  basicstyle=\ttfamily\scriptsize,
  keywordstyle=\color{blue},
  stringstyle=\color{darkgreen},
  commentstyle=\color{gray},
  extendedchars=false,
  showspaces=false,
  showstringspaces=false,
  numbers=left,
  numberstyle=\tiny,
  breaklines=true,
  backgroundcolor=\color{blue!10},
  breakautoindent=true,
  captionpos=b,
  xleftmargin=0pt,
}

% Para acrescentar comentários ao PDF descomente:
\usepackage
%  [pdfauthor={nome do autor},
%   pdftitle={titulo},
%   pdfkeywords={palavra-chave, palavra-chave},
%   pdfproducer={Latex with hyperref},
%   pdfcreator={pdflatex}]
{hyperref}

%\addbibresource{referencias.bib}

\newcommand{\fromeng}[1]{\footnote{do inglês: \textit{#1}}}
\newcommand{\tit}[1]{\textit{#1}}
\newcommand{\tbf}[1]{\textbf{#1}}
\newcommand{\ttt}[1]{\texttt{#1}}
%--------------------------------------------------------------------
%TEMPLATE ADAPTADO PARA OS TRABALHOS DO PROGRAMA DE FORMAÇÃO ANALYTICS
%Fiquem a vontade para usar esse template, ele já contém dois pacotes para a escrita de código e para formalizar algoritmos.
%Abraços! Qualquer dúvida podem me chamar no whats.
%Guilherme Cano Lopes

\begin{document}
%--------------------------------------------------------------------
%Pessoal, basta alterar as informações abaixo pra editar a capa
\autor{Nome do Aluno}
\titulo{TÍTULO DO PROJETO}

\orientador{Nome do Professor}

%(não mexer):
\mestrado

\datadadefesa{12}{09}{2018}
\paginasiniciais
%\begin{flushleft}

%Comente a linha de baixo se não desejar usar um sumário
%(remova o * dos comandos \chapter para que ele apareça no sumário)
%\tableofcontents
\fimdaspaginasiniciais
%-------------------------------------------------------------------------

\chapter*{1. Pacote para Algoritmos}
%O pacote se chama algorithm2e, consultem a documentação pra entender seu uso.
\begin{algorithm}[ht]
   \SetAlgoLined
   \Entrada{Grafo G(V,E) e coloração c (certificado).} 
   \Saida{Verdadeiro ou Falso}
   \Inicio{
    \Para{cada aresta $(u,v) \in E$}{
    \Se{c(u) = c(v)}{
    \textbf{devolva} Falso
    }
    }  
   \textbf{devolva} Verdadeiro
   }
   \label{alg2}
   \caption{\textsc{Verificador($G,c$)}
   }
\end{algorithm}

\chapter*{2. Escrita de código (configurado para Python)}
%É possível configurar a linguagem editando as configurações do pacote no início deste .tex
\begin{lstlisting}
#codigo exemplo (nao use acentuacao nos comentarios que vao pra o lstlisting)
def plot_3d(x,y,z,title=None,label="",id=0):
    mpl.rcParams['legend.fontsize'] = 10

    fig = plt.figure(id)
    ax = fig.gca(projection='3d')
    x = x.as_matrix()
    y = y.as_matrix()
    z = z.as_matrix()

    ax.plot(x, y, z, label=label)
    ax.legend(bbox_to_anchor=(0.3, 0.8))
    if title is not None:
        plt.title(title)

def main():
    plot_3d(x = data['x (Meters)'], y=data['y (Meters)'], z = data['z (Meters)'],label="Drone Position")
    
if __name__ == "__main__":
   main() 
\end{lstlisting}

Para citar trabalhos, use o comando \cite{lopes2014elastic}. 

%Referências
\newpage
\bibliographystyle{IEEEtran}
\bibliography{referencias}

\end{document}